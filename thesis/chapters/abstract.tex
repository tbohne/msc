\chapter*{Abstract}
\begin{prefacesection}
\noindent
Long-term autonomous robotic systems in highly dynamic real-world applications are becoming a reality.
Nonetheless, ensuring a certain level of robustness for these systems is a major challenge, as it is
impossible to anticipate all potential situations that a robot might encounter. A joint consistent
integration of both monitoring and acting is vital, as plans do not always work out as expected when
faced with the chaotic nature of reality. Consequently, execution monitoring techniques that address
unexpected problematic situations by means of recovery mechanisms must be an essential component of
robot architectures. This thesis attempts to reduce the barriers towards long-term autonomy of mobile
outdoor robots by identifying and classifying key difficulties in such a context and developing a fully
integrated monitoring and resolution framework capable of overcoming some of the typical limitations for
those systems. Experimental evaluation of the proposed framework in a simulation environment indicates a
drastically improved resilience with respect to the identified challenges.

\vfill

\subsubsection*{Source Code \& Documentation}
\begin{itemize}
  \item Execution Monitoring Framework: \textcolor{link-color}{\url{https://github.com/tbohne/execution_monitoring}}
  \item Plan Generation Node: \textcolor{link-color}{\url{https://github.com/tbohne/plan_generation}}
  \item Thesis (\LaTeX{}), Resources and Manual: \textcolor{link-color}{\url{https://github.com/tbohne/msc}}
\end{itemize}
{\footnotesize State at the time of the thesis submission: \code{git tag v1.0}}

\end{prefacesection}
