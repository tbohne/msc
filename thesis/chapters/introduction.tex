\chapter{Introduction}

In recent years, progressive automation has increasingly entered the field of agriculture. In particular, long-term autonomous mobile robots offer tremendous potential for
agricultural applications, as these often entail extended periods of time (seasonal cycles) as well as large areas to cover. Since the beginning of the 21st century, numerous
robotic applications for agriculture have been developed to increase productivity and improve the accuracy, efficiency and safety of agricultural production processes. \cite{Xue:2010}
Efficiency improvements in agriculture are not only sensible, but perhaps even necessary. The rapidly growing human population and environmental changes require a significant
increase in production to meet the rising demand. \cite{Virlet:2016}\cite{KhakPour:2021}\cite{Roure:2018} However, it is not the case that robots can merely improve the efficiency
in tasks that could have been done by humans. There are examples of tasks that only become possible through the use of robotic systems: \textquote{\textit{A robot's \textquote{eye}
is far better than a human one, and it can collect a large amount of data that are invisible for us.}} \cite{Ampatzidis:2017} The idea that robots can be used to efficiently monitor
large areas has numerous real-world applications in precision agriculture. \cite{Bargoti:2015} Autonomous crop monitoring, a classic domain of precision agriculture, has the goal of
detecting potential problems as soon as possible in order to avoid yield losses while optimizing costs (economic efficiency) and environmental impact. \cite{Dong:2014}
\cite{Dong:2017} The ultimate goal of precision agriculture is to automatically obtain relevant crop health information to detect and treat biotic and abiotic stresses to prevent
yield loss without the need for manual sampling, which is too expensive and inefficient. \cite{Carlone:2015}\newline
On the technological side, long-term autonomous robotic systems in highly dynamic real-world applications are becoming a reality. \cite{Kunze:2018} Nonetheless, achieving robust
long-term autonomy is particularly challenging because it is impossible to envision all potential situations that the robot may encounter. \cite{Hawes:2017}
As Ingrand at al. put it: \textquote{\textit{Autonomous robots facing a diversity of environments, a variety of tasks and a range of interactions cannot be
preprogrammed by foreseeing at the design stage all possible courses of actions they may require.}} \cite{Ingrand:2017}
Thus, while long-term autonomous robots have enormous potential to increase efficiency and reduce the burden on humans in many areas, they are often not yet reliable
and safe enough to be used with a clear conscience in a wide variety of real-world environments. \cite{Arvin:2018} Execution monitoring approaches that address failures preventing
a system's long-term autonomy by providing recovery mechanisms have to be a crucial component of robotic architectures. \cite{Kunze:2018}
Commonly, autonomous robot operations were often closely supervised by human operators, who intervened whenever the robot reached its limits. \cite{Rosenthal:2011}
Only recently has a shift begun toward systems equipped with the ability to recognize their own shortcomings and actively request help from human operators, who do not
necessarily monitor them closely, but only provide assistance when needed. \cite{Rosenthal:2011} According to Pinillos et al. there are two major requirements that robots must meet
in order to be commercially successful: A reasonable price-performance ratio and a robust task performance. \cite{Pinillos:2016} This thesis is primarily concerned with the latter.
Almost regardless of the actual service a robot provides, it must be robust in its basic autonomous capabilities such as localization, navigation, perception, etc. Pinillos et al.
underline another aspect that is a major motivation for this work: \textquote{\textit{Research efforts often focus on robot localization, navigation, planning or face recognition,
but fail in the integration of different technologies to create useful applications.}} \cite{Pinillos:2016} This is precisely the objective of this thesis - to work towards robust
integrated systems. Intriguingly, agricultural scenarios represent one of the most challenging domains in the field of autonomous mobile robotics. Although (semi-)autonomous
planetary exploration rovers such as NASA's \textit{Curiosity} encounter manifold and demanding tasks, the environment is quite predictable and exhibits
rather low semantic complexity, in contrast to agricultural robots, which face a richer environment populated by humans, etc. \cite{Ingrand:2017}\newline

\noindent
\textit{Goal}\newline

\noindent
As the title suggests, the goal of this work is to integrate a prototypical system for long-term autonomous plant observation that is able to address 
particular challenges for long-term autonomy in such a context through the use of execution monitoring methods. Initially, typical challenges that may stand in the way of a mobile
robot's long-term autonomy in an agricultural setting will be identified. Afterwards, a subset of these challenges will be studied systematically in order to 
approach robust solutions. Accordingly, the prolonged objective is to increase the number of situations in which a mobile robot is able to 
overcome such challenges by itself. However, the mere recognition of problematic situations is already of decisive advantage from a practical perspective,
as the robot is then able to communicate the problem and request help, e.g., from a human operator, instead of undeliberately aborting its mission.
In addition, a prerequisite for solving a problem is, of course, recognizing it. Therefore, a major aim of this thesis is to develop execution monitoring
approaches that enable the robot to detect unexpected problematic situations that require special treatment.
A very important aspect of working with robots in real-world practical scenarios is the merging of the worlds of abstract AI planning and execution of those plans with
the real robot, i.e., acting. As Ghallab et al. phrase it \textquote{\textit{Planning and acting require significant deliberation because an intelligent system must coordinate and
integrate these activities in order to act effectively in the real world}}. \cite{GNT:2016} There is much literature that considers abstract planning, i.e., generating high-level
task schedules, and also a number of works on integrating planning and acting. Somewhat underrepresented so far is the aspect of monitoring the execution of the generated
plans, especially in long-term scenarios, which will be the subject of this thesis. In particular, little attention has been paid to a joint consistent integration of both monitoring
and acting. \cite{Ingrand:2017} This is crucial as plans will not always work out as expected when faced with the dynamic and chaotic nature of reality.\newline

\vfill
\pagebreak

\noindent
\textit{Expected Scientific Contribution}\newline

\noindent
Key challenges that may prevent long-term autonomous mobile robots in a field monitoring context from working properly will be identified, implemented 
in a simulation and systematically evaluated. A major objective of this work is to address a subset of these challenges based on execution monitoring approaches,
i.e., to propose methods that are capable of partially resolving or at least detecting such issues in order to enable the robot to communicate them to a human operator.
If it is possible to show that with the solutions developed in this work, practically relevant problems can be reliably detected, at least communicated, and possibly even solved
by the robot itself, the system can be considered a relevant step towards the overall goal of robust, long-term autonomous field monitoring robots. In order to provide not only a
tailored solution for a specific scenario, as few assumptions as possible are made about specific systems, implying that all developed solutions should be generalizable to other
systems as well. Sensors, for instance, are regarded as black boxes. There will be some assumptions about generic data structures and information required for a concrete solution,
but the claim is to solve general problems of outdoor robots and abstract from specific sensor models, etc.
 In summary, this work attempts to reduce the barriers towards long-term autonomy of mobile outdoor robots by demonstrating a fully integrated solution capable of
addressing some of the typical problems for such systems. For this purpose, an execution monitoring architecture is designed and implemented. Although the work deals specifically with
an agricultural context, the approaches are developed with the aspiration of a certain universality. A thorough literature search revealed that such a system, which attempts to
systematically address common problems for long-term autonomous mobile outdoor robots, does not appear to exist yet, making it a meaningful contribution. Already the compilation of
a list of common problems in such contexts is valuable, as such a compilation and joint consideration of these problems does not yet seem to exist in the literature.\newline

\noindent
\textit{Approach}\newline

\noindent
The idea is to start with the basic long-term autonomy setup described in section \ref{sec:prototype_scenario}, provide a list of potential problems that 
stand in its way, and develop execution monitoring methods to detect a subset of these issues with the aim of increasing the robustness of such a system in a simulation
and perspectively in practice on the real hardware. Before delving into the details, it is essential to define what is actually meant by the term \textit{robust autonomy} in the
context of this thesis. Brodskiy et al. sum it up precisely: \textquote{\textit{[...] robust autonomy is mostly considered as ability of the system to react to changes in the
environment, unexpected situations or special conditions. The phenomena to which the system is supposed to react can be summarised as abnormal events, happenings outside the
normal workflow. Without special considerations the abnormal events become the cause of the system failure.}}. \cite{Brodskiy:2011}
In the end, there should be a working system that addresses some of the problems of long-term autonomy
in scenarios similar to the one described in section \ref{sec:lta_plant_observation}. The nature of the work is going to be integrative and application-oriented. The modules 
needed to set up the initial prototype are available in principle, but it will be part of this work to integrate and extend them in order to end up with a holistic and robust
solution. After the basic scenario works in the simulation, it is going to be extended with an evaluation part. A subset of the potential barriers described in section 
\ref{sec:challenges_for_lta} is going to be implemented and it will be shown that the system no longer works under certain conditions, i.e., that these
barriers are indeed able to cause failure of long-term autonomous systems. Consequently, a first step is to observe how the system performs without any further 
treatment of such problematic situations.
Subsequently, the goal is to develop monitoring methods capable of recognizing these problems such that they can be resolved.
To give an example, a particular idea could be to block paths on the field in the simulation in a randomized fashion with the intent of reproducibly making things that 
can go wrong actually go wrong.
It is obvious that the robot must be enabled to detect such problems, in this case by some kind of obstacle detection.
Once detected, the problem can be resolved by either incorporating solutions from the literature or finding new solutions.
Since detection is a necessary prerequisite for overcoming these barriers, the focus will be on detection, i.e., execution monitoring,
and an initial trivial solution adopted for all of them is to call the human operator who then takes care of the problem.\newline
As should be clear by now, long-term autonomy, as robotics in general, is an integration problem. Many technologies have to be integrated in order to build a working system, 
and compared to other disciplines of artificial intelligence, it is not trivial to evaluate the system and conclude, for example, that it has improved the status quo by a 
certain percentage. Nevertheless, it is crucial to provide results that are meaningful based on scientific standards.
Since it is beyond the scope of this work to test the developed system for extended periods of time in practice, there is a need for other ways of evaluation.
This is where the described evaluation approach in the simulation comes into play. In order to demonstrate, test and evaluate the system, there will be a sufficiently meaningful, 
i.e., realistic, physics simulation that allows an empirical analysis. Ultimately, there should also be a long-term test (e.g. one day) in the field with the real hardware that 
underlines the relevance of the approaches discussed in this work for practical applications. However, since a mobile robot is a complex system, there are arbitrary 
many technical barriers that could prevent such a long-term test in the real world. The idea is therefore to be at least able to demonstrate the system in the simulation
without depending on the success of such a real-world experiment. In addition, a simulation naturally creates an environment that allows reproducible situations and thus permits
empirical investigation.\newline
Finally, it is essential to note that the robotics framework used in this thesis is the \textit{Robot Operating System} (ROS) \cite{Quigley:2009}.
All software developed as part of this work follows ROS paradigms and operates in the ROS ecosystem.\newline

\noindent
Initially, chapter \ref{sec:lta_mobile_robots} provides a background on long-term autonomous mobile robots, i.e., a literature review, the scenario considered in this thesis,
the technological basis, as well as the challenges for long-term autonomy in such a context. This is followed by chapter \ref{sec:plan_execution_and_monitoring} on the eponymous
execution monitoring, which aims to identify and resolve these challenges. Subsequently, chapter \ref{sec:integrated_solutions} describes the integrated solutions that enrich the
functionality of the system. Eventually, chapter \ref{sec:experiments} covers the experimental evaluation and chapter \ref{sec:conclusion_future_work} concludes the thesis with
final remarks, discussion and future work.
