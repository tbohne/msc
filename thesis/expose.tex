% class options:
% - select either [german] or [english]
% - select the type of thesis from:
%   [bachelor, master, generic]
%   (in case of generic, use \type{} to specify it)
% - use option "alpha" for abbreviated citation (instead of numbers)
% - option "draft" is available, too
% - use options "utf8" or "latin1" to select inputencoding
\documentclass[german, master, expose, latin1]{base/thesis_KBS}

\usepackage{units}    % useful for settings units:              \unit[23]{m}
\usepackage{nicefrac} % for setting fractions esp. within text: \nicefrac{km}{h}

\usepackage{algorithm, algorithmic}  % for pseudo code (cf. documentation)
\renewcommand{\algorithmiccomment}[1]{\qquad{\small // \textit{#1}}}

%%%%%%%%%%%%%%%%%%%%%%%%%%%%%%%%%%%%%%%%%%%%%%%%%%%%%%%%%%%%%%%%%%%%%%%%%%%%%%%

\begin{document}

\title{Working Title: Long-Term Autonomous Plant Monitoring with a Mobile Robot}
\author{Tim Bohne}
\email{tbohne@uni-osnabrueck.de}

\generatetitle

\section{Goal}

As the title suggests, the goal of the work is to develop a prototypical system for long-term autonomous plant monitoring with a mobile robot.\newline

The first essential step towards long-term autonomy of a mobile robot is of course to ensure its power supply.
For this purpose, there is an inductive charging station, which is located in a mobile container on the field site.
Thus, the power supply is realized by an autonomous docking / undocking solution, which enables the robot to detect the container in a laser scan
of its nearby surroundings, drive into it and dock to an inductive charging station. To know when to trigger this docking procedure, it is crucial 
to have a process that monitors the battery level. Based on the dimensions of the particular field under consideration, it is not assured that one battery 
charge will be sufficient to process every parcel.\newline

When the power supply is established, the robot's actual task can be performed. This task consists of planning when to observe which parcel of the field,
how to get there, and what sensors to use. Based on the growth state of the plants, different sensors are required to record the relevant data.
Just as important as the planning itself, is the execution of the generated plan as well as monitoring the execution.
So, in general this will be a task of plan-based robot control.
Although the overall tour, the robot is going to drive through is generally always the same, there are certain stopping conditions 
such as a low battery level or drastic weather changes that lead to a preemption of an active scanning tour and require the robot 
to move back to its base, i.e. the mobile container. Afterwards, when the cause of the preemption is resolved, 
the robot should continue its scanning task right from the point where it was preempted.\newline

The plan, i.e. the decision of which parcel is to be processed next, is based on a list of criteria. The primary criterion is the growth stage of the plants.
There is a list of particular features that are supposed to be detected based on the current growth state.
At the moment, this information is communicated manually, but in the future it is planned to have that automated.
It is well traceable in time when a certain feature is to be expected such that this process could be time-triggered.\newline

Finally, the goal is to generate a complete picture of every parcel on the field, not randomly, but time targeted and based on weather conditions. 
Especially when considering hyperspectral data, it is important that it is neither too sunny, nor too cloudy, ideally it should be grey and foggy with uniform illumination.
Of course, it's not trivial to predict the weather.\newline

Initially, a prototypical setup of a long-term autonomous plant monitoring scenario will be developed that can be tested and evaluated 
in a physics simulation, i.e. Gazebo. Ultimately, there should be a long-term experiment (e.g. one day) in the real world with the actual hardware that
underlines the relevance of the approaches discussed in this work for practical applications. Since there are arbitrary many technical barriers that 
could prevent a practical success in a long-term test, the idea is to be at least able to show it in a sufficiently realistic physical simulation.\newline

The nature of the work is going to be integrative and application-oriented, many modules that are needed to solve the task at hand are available in principle, 
but it will be the goal of this work to identify those, evaluate them empirically for their particular part of the task and integrate them in a wholesome 
solution that is capable of providing an example of a robust solution for practical applications.\newline

\section{Scientific and/or technological background}

\textbf{Challenges}\newline

Based on the growth state of the plants, the planning has to be adapted, which has to be evaluated in the simulation, because a long-term test that reflects
seasonal changes would clearly be beyond the scope of this work. So, besides the need for a simulation in terms of potential hardware-related or other hurdles
that prevent practical testing, we have another motivation for the simulation, which are seasonal changes that are needed in order to evaluate the different
planning strategies. From a certain date, the plants are too high to drive above them and the scans have to be taken from a different spot, the whole strategy changes.
When the plants are small enough, the scans are taken from above, when they become too big, that's no longer possible.
That is a typical constraint in agricultural projects - the vegetation periods.
In March, when this work is expected to end, the plants are going to be relatively small such that the robot can drive above them when taking
scans, in the end of srping and summer that would no longer be possible and will be reflected in the simuation.

\section{Approach}

The plan is to start from the goal backwards with the question of what is required in order to end up with a long-term autonomous system that is
able to perform those scans with changing environment conditions etc. in the simulation and ideally in practice on the real hardware.
A physica simulation in Gazebo is created as part of the research project PORTAL and might be exteneded as part of this work, e.g. with weather conditions.

\section{Expected scientific and/or technological contribution}

The idea of the long-term test is that the robot has to drive back to its charging station at least two times, otherwise it would be vacuous.
If possible, it would be good to include real weather data form the region of the test field.

\section{Work program}

TBD.

% DON'T set \bibliographystyle here -- use the documentclass option instead
\bibliography{papers}

\end{document}
